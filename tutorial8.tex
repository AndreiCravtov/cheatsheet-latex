\subsection*{Eigenvalue Problems: Iterative
  Techniques}

\begin{itemize}

  \item
        If \iMbox{A} is {[}{[}tutorial 1\#Properties of
        matrices\textbar diagonalizable{]}{]} then {[}{[}tutorial
        1\#Eigen-values/vectors\textbar eigen-decomposition{]}{]}
        \iMbox{A = X \Lambda X^{-1}}

        \begin{itemize}

          \item
                \textbf{Dominant} \iMbox{\lambda_{1};\mathbf{x}_{1}} are such that
                \iMbox{\lvert \lambda_{1} \rvert} is strictly largest for which
                \iMbox{A\mathbf{x} = \lambda \mathbf{x}}
          \item
                \textbf{Rayleigh quotient} for Hermitian \iMbox{A = A ^{\dagger}} is
                \iMbox{\ds R_A(\mathbf{x}) = \frac{\mathbf{x}^{\dagger}A\mathbf{x}}{\mathbf{x}^{\dagger}\mathbf{x}}}

                \begin{itemize}

                  \item
                        Eigenvectors are stationary points of \iMbox{\ds R_{A}}
                  \item
                        \iMbox{\ds R_{A}(\mathbf{x})} is closest to being like eigenvalue
                        of \iMbox{\mathbf{x}},
                        i.e.~\iMbox{\ds R_{A}(\mathbf{x})=\underset{\alpha}{\operatorname{argmin}}\|A \mathbf{x}-\alpha \mathbf{x}\|_2}
                  \item
                        \iMbox{\ds R_{A}(\mathbf{x}) - R_{A}(\nu) = O(\lVert \mathbf{x} - \nu \rVert^{2})}
                        as \iMbox{\mathbf{x} \to \nu} where \iMbox{\nu} is eigenvector
                \end{itemize}
        \end{itemize}
  \item
        \textbf{Power iteration}: define sequence
        \iMbox{\ds \mathbf{b}^{(k+1)}=\frac{A \mathbf{b}^{(k)}}{\left\|A \mathbf{b}^{(k)}\right\|}}
        with initial \iMbox{\ds\mathbf{b}^{(0)}} s.t.
        \iMbox{\lVert \mathbf{b}^{(0)} \rVert = 1}

        \begin{itemize}

          \item
                Assume \textbf{dominant} \iMbox{\lambda_{1};\mathbf{x}_{1}} exist
                for \iMbox{A}, and that
                \iMbox{\ds \mathrm{proj}_{\ds\mathbf{x}_{1}}\left(\ds\mathbf{b}^{(0)} \right) \neq \mathbf{0}}
          \item
                Under above assumptions,
                \iMbox{\ds \mu_{k} = R_A\left(\mathbf{b}^{(k)} \right) = \frac{{\mathbf{b}^{(k)}}^{\dagger}A\mathbf{b}^{(k)}}{{\mathbf{b}^{(k)}}^{\dagger}\mathbf{b}^{(k)}}}
                converges to \textbf{dominant \iMbox{\lambda_{1}}}
          \item
                \iMbox{\ds\langle\mathbf{b}_{k}\rangle} converges to some
                \textbf{dominant} \iMbox{\mathbf{x}_{1}} associated with
                \iMbox{\lambda_{1}} =\textgreater{}
                \iMbox{\ds \left\|A \mathbf{b}^{(k)}\right\|} converges to
                \iMbox{|\lambda_{1}|}
          \item
                If
                \iMbox{\ds \mathrm{proj}_{\ds\mathbf{x}_{1}}\left(\ds\mathbf{b}^{(0)} \right) =\mathbf{0}}
                then
                \iMbox{\ds\langle\mathbf{b}_{k}\rangle;\langle\mathbf{\mu}_{k}\rangle}
                converge to second \textbf{dominant}
                \iMbox{\lambda_{2};\mathbf{x}_{2}} instead
          \item
                If \textbf{no dominant \iMbox{\lambda}} \emph{(i.e.~multiple
                  eigenvalues of maximum \iMbox{|\lambda|})} then
                \iMbox{\ds\langle\mathbf{b}_{k}\rangle} will converge to linear
                combination of their corresponding eigenvectors
          \item
                Slow convergence if \textbf{dominant \iMbox{\lambda_{1}}} not ``very
                dominant''
          \item
                \iMbox{\ds \lVert \mathbf{b}^{(k)} - \alpha_{k}\mathbf{x}_{1} \rVert = O\left( \left\lvert  \frac{\lambda_{2}}{\lambda_{1}}  \right\rvert^k \right)}
                for \textbf{phase factor} \iMbox{\alpha_{k} \in \{ -1,1 \}} it may
                alternate if \iMbox{\lambda_{1}<0}

                \begin{itemize}

                  \item
                        \iMbox{\ds \alpha_{k} = \frac{(\lambda_{1})^k c_{1}}{|\lambda_{1}|^k |c_{1}|}}
                        where \iMbox{\ds c_{1} = \mathbf{x}_{1}^{\dagger}\mathbf{b}^{(0)}}
                        and assuming \iMbox{\mathbf{b}^{(k)};\mathbf{x}_{1}} are
                        normalized
                \end{itemize}
          \item
                \iMbox{(A - \sigma I)} has \textbf{eigenvalues}
                \iMbox{\lambda-\sigma} =\textgreater{} power-iteration on
                \iMbox{(A - \sigma I)} has
                \iMbox{\ds\frac{\lambda_{2}-\sigma}{\lambda_{1}-\sigma}}
          \item
                Eigenvector guess =\textgreater{} estimated eigenvalue
        \end{itemize}
  \item
        \textbf{Inverse (power-)iteration}: perform power iteration on
        \iMbox{\ds (A - \sigma I)^{-1}} to get \iMbox{\ds\lambda_{1,\sigma}}
        \textbf{closest} to \iMbox{\sigma}

        \begin{itemize}

          \item
                \iMbox{(A - \sigma I)^{-1}} has eigenvalues
                \iMbox{(\lambda- \sigma)^{-1}} so power iteration will yield
                \textbf{largest \iMbox{(\lambda_{1,\sigma}- \sigma)^{-1}}}
          \item
                i.e.~will yield \textbf{smallest
                  \iMbox{\lambda_{1,\sigma}- \sigma}}, i.e.~will yield
                \iMbox{\ds\lambda_{1,\sigma}} \textbf{closest} to \iMbox{\sigma}
          \item
                \iMbox{\ds \lVert \mathbf{b}^{(k)} - \alpha_{k}\mathbf{x}_{1,\sigma} \rVert = O\left( \left\lvert  \frac{\lambda_{1,\sigma} - \sigma}{\lambda_{2,\sigma} - \sigma}  \right\rvert^k \right)}
                where \iMbox{\ds\mathbf{x}_{1,\sigma}} corresponds to
                \iMbox{\ds\lambda_{1,\sigma}} and \iMbox{\ds \lambda_{2,\sigma}} is
                2nd-closest to \iMbox{\sigma}
          \item
                Efficiently compute eigenvectors for \textbf{known eigenvalues}
                \iMbox{\sigma}
          \item
                Eigenvalue guess =\textgreater{} estimated eigenvector
          \item
                !{[}{[}Pasted image 20250420131643.png\textbar300{]}{]}
          \item
                Can reduce matrix inversion \iMbox{O(m^3)} to \iMbox{O(m^2)} by
                pre-factorization
        \end{itemize}
\end{itemize}

\subsection*{Nonlinear Systems of Equations: Iterative
  Techniques}

\begin{itemize}

  \item
        {[}{[}tutorial 6\#Multivariate Calculus\textbar Recall{]}{]} that
        \iMbox{\ds \nabla f(\mathbf{x})} is direction of \textbf{max.}
        rate-of-change \iMbox{\ds \lvert \nabla f(\mathbf{x}) \rvert}
  \item
        Search for stationary point by \textbf{gradient descent}:
        \iMbox{\ds \mathbf{x}^{(k+1)} = \mathbf{x}^{(k)} - \alpha \nabla f(\mathbf{x}^{(k)})}
        for step length \iMbox{\alpha}
  \item
        \iMbox{A} is positive-definite solving
        \iMbox{A\mathbf{x} = \mathbf{b}} and
        \iMbox{\ds \min_{\mathbf{x}} f(\mathbf{x}) = \frac{1}{2} \mathbf{x}^T A \mathbf{x} - \mathbf{x}^T \mathbf{b}}
        are equivalent

        \begin{itemize}

          \item
                Get iterative methods
                \iMbox{\ds \mathbf{x}^{(k+1)} = \mathbf{x}^{(k)} - \alpha^{(k)} \mathbf{p}^{(k)}}
                for step length \iMbox{\ds\alpha^{(k)}} and directions
                \iMbox{\ds\mathbf{p}^{(k)}}
        \end{itemize}
  \item
        \textbf{Conjugate gradient (CG) method}: if
        \iMbox{A \in \mathbb{R}^{n \times n}} also symmetric then
        \iMbox{\ds \langle \mathbf{u},\mathbf{v} \rangle_{A} = \mathbf{u}^T A\mathbf{v}}
        is an inner-product

        \begin{itemize}

          \item
                \textbf{GC} chooses \iMbox{\ds \mathbf{p}^{(k)}} that are conjugate
                w.r.t. \iMbox{A},
                i.e.~\iMbox{\ds \langle \mathbf{p}^{(i)},\mathbf{p}^{(j)} \rangle_{A} = 0}
                for \iMbox{i\neq j}
          \item
                And chooses \iMbox{\alpha^{(k)}} s.t. \textbf{residuals}
                \iMbox{\ds \mathbf{r}^{(k)} = - {\nabla f}(\mathbf{x}^{(k)}) = \mathbf{b} - A\mathbf{x}^{(k)}}
                are orthogonal

                \begin{itemize}

                  \item
                        \iMbox{k=0} =\textgreater{}
                        \iMbox{\ds \mathbf{p}^{(0)} = - {\nabla f}(\mathbf{x}^{(0)}) = \mathbf{r}^{(0)}}
                  \item
                        \iMbox{k\geq 1} =\textgreater{}
                        \iMbox{\ds \mathbf{p}^{(k)} = \mathbf{r}^{(k)} - \sum_{i<k} {\frac{\langle \mathbf{p}^{(i)},\mathbf{r}^{(k)} \rangle_{A}}{\langle \mathbf{p}^{(i)},\mathbf{p}^{(i)} \rangle_{A}}\mathbf{p}^{(i)}}}
                  \item
                        \iMbox{\ds \alpha^{(k)} = \mathrm{argmin}_{\alpha} {f(\mathbf{x}^{(k)} + \alpha^{(k)} \mathbf{p}^{(k)})} = \frac{\mathbf{p}^{(k)} \cdot \mathbf{r}^{(k)}}{\langle \mathbf{p}^{(k)},\mathbf{p}^{(k)} \rangle_{A}}}
                \end{itemize}
          \item
                Without rounding errors, \textbf{CG} converges in \iMbox{\leq n}
                iterations

                \begin{itemize}

                  \item
                        Similar to to {[}{[}tutorial 1\#Gram-Schmidt method to generate
                        orthonormal basis from any linearly independent
                        vectors\textbar Gram-Schmidt{]}{]} \emph{(different
                          inner-product)}
                  \item
                        \iMbox{\ds\langle \mathbf{p}^{(0)},\dots,\mathbf{p}^{(n-1)} \rangle}
                        and
                        \iMbox{\ds\langle \mathbf{r}^{(0)},\dots,\mathbf{r}^{(n-1)} \rangle}
                        are bases for \iMbox{\mathbb{R}^{n}}
                \end{itemize}
        \end{itemize}
\end{itemize}

\subsection*{\texorpdfstring{QR Algorithm to find Schur decomposition
    \iMbox{A = QUQ^{\dagger}}}{QR Algorithm to find Schur decomposition }}

\begin{itemize}

  \item
        Any \iMbox{A \in \mathbb{C}^{m \times m}} has \textbf{Schur
          decomposition} \iMbox{A = QUQ^{\dagger}}

        \begin{itemize}

          \item
                \iMbox{Q} is unitary, i.e.~\iMbox{Q^{\dagger} = Q^{-1}} and
                upper-triangular \iMbox{U}
          \item
                Diagonal of \iMbox{U} contains \textbf{eigenvalues} of \iMbox{A}
        \end{itemize}
  \item
        !{[}{[}Pasted image 20250420135506.png\textbar300{]}{]}
  \item
        For \iMbox{A \in \mathbb{R}^{m \times m}} each iteration
        \iMbox{\ds A^{(k)} = Q^{(k)}R^{(k)}} produces orthogonal
        \iMbox{{Q^{(k)}}^{T} = {Q^{(k)}}^{-1}}
  \item
        So
        \iMbox{\ds A^{(k+1)}=R^{(k)} Q^{(k)}= ({Q^{(k)}}^{T} Q^{(k)}) R^{(k)} Q^{(k)}={Q^{(k)}}^{T} A^{(k)} Q^{(k)}}
        means \iMbox{A^{(k+1)}} is \textbf{similar} to \iMbox{A^{(k)}}

        \begin{itemize}

          \item
                Setting \iMbox{\ds A^{(0)} = A} we get
                \iMbox{A^{(k)} = {\tilde{Q}^{(k)}}^{T} A \tilde{Q}^{(k)}} where
                \iMbox{\ds\tilde{Q}^{(k)} = Q^{(0)}\cdots Q^{(k-1)}}
        \end{itemize}
  \item
        Under certain conditions \textbf{QR algorithm} converges to
        \textbf{Schur decomposition}
  \item
        We can \textbf{apply shift} \iMbox{\mu^{(k)}} at iteration \iMbox{k}
        =\textgreater{}
        \iMbox{\ds A^{(k)} - \mu^{(k)}I = Q^{(k)}R^{(k)}; \ A^{(k+1)} = R^{(k)}Q^{(k)} + \mu^{(k)}I}

        \begin{itemize}

          \item
                If \textbf{shifts} are good eigenvalue estimates then last column of
                \iMbox{\tilde{Q}^{(k)}} converges quickly to an \textbf{eigenvector}
          \item
                Estimate \iMbox{\mu^{(k)}} with Rayleigh quotient =\textgreater{}
                \iMbox{\ds \mu^{(k)} = (A_{k})_{mm} = {\tilde{\mathbf{q}}^{(k)}_{m}}^{T} A \tilde{\mathbf{q}}^{(k)}_{m}}
                where \iMbox{\tilde{\mathbf{q}}^{(k)}_{m}} is \iMbox{m}-th column of
                \iMbox{\tilde{Q}^{(k)}}
        \end{itemize}
\end{itemize}