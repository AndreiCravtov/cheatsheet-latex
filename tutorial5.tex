\subsection*{Thin QR Decomposition w/ Gram-Schmidt (GS)}


Consider \textbf{full-rank}
\iMbox{\ds A = [\mathbf{a}_{1}|\dots|\mathbf{a}_{n}] \in \mathbb{R}^{m \times n}}
\emph{(\iMbox{m \geq n})},
i.e. \iMbox{\ds\mathbf{a}_{1},\dots,\mathbf{a}_{n} \in \mathbb{R}^{m}}
are linearly independent

\begin{itemize}

      \vItem
            Apply \ul{GS} \iMbox{\ds\mathbf{q}_{1},\dots,\mathbf{q}_{n} \leftarrow \mathrm{GS}(\mathbf{a}_{1},\dots,\mathbf{a}_{n})}
            to build \textbf{ONB}
            \iMbox{\ds\langle \mathbf{q}_{1},\dots,\mathbf{q}_{n} \rangle \in \mathbb{R}^m}
            for \iMbox{\mathrm{C}(A)}
      \vItem
            \textbf{For exams}: more efficient to compute as
            \iMbox{\ds \mathbf{u}_{j+1} = \mathbf{a}_{j+1} - Q_{j}\mathbf{c}_{j}}

            \begin{enumerate}
                  \vItem 1)
                        Gather
                        \iMbox{\ds Q_{j} = [\mathbf{q}_{1}|\dots|\mathbf{q}_{j}] \in \mathbb{R}^{m \times j}}
                        \textbf{all-at-once}
                  \vItem 2)
                        Compute
                        \iMbox{\ds \mathbf{c}_{j} = [\mathbf{q}_{1} \cdot \mathbf{a}_{j+1}, \dots, \mathbf{q}_{j} \cdot \mathbf{a}_{j+1}]^{T} \in \mathbb{R}^{j}}
                        \textbf{all-at-once}
                  \vItem 3)
                        Compute \iMbox{\ds Q_{j}\mathbf{c}_{j} \in \mathbb{R}^{m}}, and
                        subtract from \iMbox{\mathbf{a}_{j+1}} \textbf{all-at-once}
            \end{enumerate}
      \vItem
            Can now rewrite
            \iMbox{\mathbf{a}_{j}= \sum_{i=1}^{j} (\mathbf{q}_{i} \cdot \mathbf{a}_{j})\mathbf{q}_{i} = \mathrm{Q}_{j}\mathbf{c}_{j}}
\end{itemize}

\hSep % ----

Choose
\iMbox{\ds \mathbf{Q} = Q_{n} = [\mathbf{q}_{1}|\dots|\mathbf{q}_{n}] \in \mathbb{R}^{m \times n}},
notice its \ul{semi-orthogonal} since
\iMbox{\ds \mathbf{Q}^{T} \mathbf{Q} = \mathbf{I}_{n}}

\begin{itemize}

      \vItem
            Notice =>
            \iMbox{\ds\mathbf{a}_{j} = Q_{j} \mathbf{c}_{j} = \mathbf{Q}[\mathbf{q}_{1} \cdot \mathbf{a}_{j}, \dots, \mathbf{q}_{j} \cdot \mathbf{a}_{j}, 0, \dots, 0 ]^{T} = \mathbf{Q} \mathrm{r}_{j}}
      \vItem
            Let
            \iMbox{R = [\mathrm{r}_{1}|\dots|\mathrm{r}_{n}] \in \mathbb{R}^{n \times n}}
            =>
            \iMbox{A = \mathbf{Q}R = \mathbf{Q} \begin{bmatrix}\mathbf{q}_{1}^{T} \mathbf{a}_{1} & \dots & \mathbf{q}_{1}^{T} \mathbf{a}_{n} \\ & \ddots & \vdots \\ 0 & & \mathbf{q}_{n}^{T} \mathbf{a}_{n}\end{bmatrix}},
            notice its \ul{upper-triangular}
\end{itemize}


\subsection*{Full QR Decomposition}

\begin{itemize}

      \vItem
            Consider \textbf{full-rank}
            \iMbox{\ds A = [\mathbf{a}_{1}|\dots|\mathbf{a}_{n}] \in \mathbb{R}^{m \times n}}
            \emph{(\iMbox{m \geq n})},
            i.e.~\iMbox{\ds\mathbf{a}_{1},\dots,\mathbf{a}_{n} \in \mathbb{R}^{m}}
            are linearly independent
      \vItem
            Apply \ul{QR decomposition} to obtain:

            \begin{itemize}

                  \vItem
                        ONB
                        \iMbox{\ds\langle \mathbf{q}_{1},\dots,\mathbf{q}_{n} \rangle \in \mathbb{R}^m}
                        for \iMbox{\mathrm{C}(A)}
                  \vItem
                        Semi-orthogonal
                        \iMbox{\ds Q_{1}= [\mathbf{q}_{1}|\dots|\mathbf{q}_{n}] \in \mathbb{R}^{m \times n}}
                        and upper-triangular \iMbox{R_{1} \in \mathbb{R}^{n \times n}},
                        where \iMbox{A = Q_{1}R_{1}}
            \end{itemize}
      \vItem
            \ul{Compute basis extension} to obtain remaining
            \iMbox{\mathbf{q}_{n+1},\dots,\mathbf{q}_{m} \in \mathbb{R}^{m}},
            where \iMbox{\ds\langle \mathbf{q}_{1},\dots,\mathbf{q}_{m} \rangle}
            is \textbf{ONB} for \iMbox{\mathbb{R}^{m}}

            \begin{itemize}

                  \vItem
                        Notice
                        \iMbox{\ds\langle \mathbf{q}_{n+1},\dots,\mathbf{q}_{m} \rangle} is
                        \textbf{ONB} for \iMbox{\mathrm{C}(A)^{\perp} = \ker(A^{T})}
                  \vItem
                        Let
                        \iMbox{Q_{2} = [\mathbf{q}_{n+1}|\dots|\mathbf{q}_{m}] \in \mathbb{R}^{m \times (m-n)}},
                        let \iMbox{\ds Q = [Q_{1}|Q_{2}] \in \mathbb{R}^{m \times m}}, let
                        \iMbox{R = [R_{1}; \mathbf{0}_{m-n}] \in \mathbb{R}^{m \times n}}
            \end{itemize}
      \vItem
            Then \textbf{full QR decomposition} is
            \iMbox{A = QR = [Q_{1}|Q_{2}]\begin{bmatrix}R_{1}\\ \mathbf{0}_{m-n}\end{bmatrix} = Q_{1}R_{1}}

            \begin{itemize}

                  \vItem
                        \iMbox{Q} is \textbf{orthogonal}, i.e. \iMbox{Q^{-1} = Q^{T}}, so
                        its a basis transformation
                  \vItem
                        \iMbox{\ds \mathrm{proj}_{\mathrm{C}(A)} = Q_{1}Q_{1}^{T}},
                        \iMbox{\ds \mathrm{proj}_{\mathrm{C}(A)^{\perp}} = Q_{2}Q_{2}^{T}}
                        are \ul{orthogonal projections} \textbf{onto} \iMbox{C(A)},
                        \iMbox{\mathrm{C}(A)^{\perp} = \ker(A^{T})} \emph{respectively}
                  \vItem
                        Notice:
                        \iMbox{\ds QQ^T = \mathbf{I}_{m} = Q_{1}Q_{1}^{T} + Q_{2}Q_{2}^{T}}
            \end{itemize}
      \vItem
            \textbf{Generalizable} to \iMbox{A \in \mathbb{C}^{m \times n}} by
            changing transpose to conjugate-transpose
\end{itemize}

\subsection*{\texorpdfstring{Lines and hyperplanes in \iMbox{\mathbb{E}^{n}({=}\mathbb{R}^{n})}}{Lines and hyperplanes in}}

Consider \textbf{standard Euclidean space}
\iMbox{\mathbb{E}^{n}({=}\mathbb{R}^{n})}

\begin{itemize}

      \vItem
            with standard basis
            \iMbox{\ds\langle \mathbf{e}_{1},\dots,\mathbf{e}_{n} \rangle \in \mathbb{R}^n}
      \vItem
            with standard origin \iMbox{\mathbf{0} \in \mathbb{R}^{n}}
\end{itemize}

\hSep % ---

A \textbf{line} \iMbox{L = \mathbb{R}\mathbf{n} + \mathbf{c}} is
\emph{characterized} by direction \iMbox{\mathbf{n} \in \mathbb{R}^{n}}
\emph{(\iMbox{\mathbf{n} \neq \mathbf{0}})} and offset from origin \iMbox{\mathbf{c} \in L}

\begin{itemize}

      \vItem
            It is customary that:

            \begin{itemize}

                  \vItem
                        \iMbox{\mathbf{n}} is a \textbf{unit vector},
                        i.e. \iMbox{\ds \lVert \mathbf{n} \rVert = \lVert \widehat{\mathbf{n}} \rVert = 1}
                  \vItem
                        \iMbox{\mathbf{c} \in L} is \textbf{closest point to origin},
                        i.e. \iMbox{\mathbf{c} \perp \mathbf{n}}
            \end{itemize}
      \vItem
            If \iMbox{\ds \mathbf{c} \neq \lambda \mathbf{n}} =>
            \iMbox{L} \textbf{not} vector-subspace of \iMbox{\mathbb{R}^{n}}

            \begin{itemize}

                  \vItem
                        i.e. \iMbox{\mathbf{0} \not\in L}, i.e.~\iMbox{L} doesn't go through the origin
                  \vItem
                        \iMbox{L} \textbf{is} affine-subspace of \iMbox{\mathbb{R}^{n}}
            \end{itemize}
      \vItem
            If \iMbox{\mathbf{c} = \lambda \mathbf{n}},
            i.e. \iMbox{\ds L = \mathbb{R} \mathbf{n}} =>
            \iMbox{\ds L} \textbf{is} vector-subspace of \iMbox{\mathbb{R}^{n}}

            \begin{itemize}

                  \vItem
                        i.e.~\iMbox{\mathbf{0} \in L}, i.e. \iMbox{L} goes through the
                        origin
                  \vItem
                        \iMbox{L} has \iMbox{\dim(L) = 1} and orthonormal basis (ONB)
                        \iMbox{\ds\{ \ \widehat{\mathbf{n}} \ \}}
            \end{itemize}
\end{itemize}

\hSep % ---

A \textbf{hyperplane} \iMbox{\begin{aligned}
      P = (\mathbb{R}\mathbf{n})^{\bot} + \mathbf{c} &= \left\{ \ x + \mathbf{c} \ \middle| \ x \in \mathbb{R}^{n}, x \perp \mathbf{n} \ \right\} \\ 
      &= \left\{ \ x \in \mathbb{R}^{n} \ \middle| \ x \cdot \mathbf{n} = \mathbf{c} \cdot \mathbf{n} \ \right\}
\end{aligned}} is \emph{characterized} by normal \iMbox{\mathbf{n} \in \mathbb{R}^{n}}
\emph{(\iMbox{\mathbf{n} \neq \mathbf{0}})} and offset from origin \iMbox{\mathbf{c} \in P}

\begin{itemize}

      \vItem
            It \emph{represents} an \textbf{\iMbox{(n-1)}-dimensional slice} of
            the \textbf{\iMbox{n}-dimensional space}
      \vItem
            It is customary that:

            \begin{itemize}

                  \vItem
                        \iMbox{\mathbf{n}} is a \textbf{unit vector},
                        i.e. \iMbox{\ds \lVert \mathbf{n} \rVert = \lVert \widehat{\mathbf{n}} \rVert = 1}
                  \vItem
                        \iMbox{\mathbf{c} \in P} is \textbf{closest point to origin},
                        i.e. \iMbox{\mathbf{c} = \lambda \mathbf{n}}
                  \vItem
                        With those =>
                        \iMbox{P = \left\{ \ x \in \mathbb{R}^{n} \ \middle| \ x \cdot \mathbf{n} = \lambda \ \right\}}
            \end{itemize}
      \vItem
            If \iMbox{\mathbf{c} \cdot \mathbf{n} \neq \mathbf{0}}
            => \iMbox{P} \textbf{not} vector-subspace of \iMbox{\mathbb{R}^{n}}

            \begin{itemize}

                  \vItem
                        i.e. \iMbox{\mathbf{0} \not\in P}, i.e. \iMbox{P} doesn't go through the origin
                  \vItem
                        \iMbox{P} \textbf{is} affine-subspace of \iMbox{\mathbb{R}^{n}}
            \end{itemize}
      \vItem
            If \iMbox{\mathbf{c} \cdot \mathbf{n} = \mathbf{0}}, i.e. \iMbox{\ds P = (\mathbb{R}\mathbf{n})^{\bot}} =>
            \iMbox{\ds P} \textbf{is} vector-subspace of \iMbox{\mathbb{R}^{n}}

            \begin{itemize}

                  \vItem
                        i.e. \iMbox{\mathbf{0} \in P}, i.e. \iMbox{P} goes through the origin
                  \vItem
                        \iMbox{P} has \iMbox{\dim(P) = n-1}
            \end{itemize}
\end{itemize}

\hSep % ---

Notice \iMbox{L = \mathbb{R}\mathbf{n}} and \iMbox{P = (\mathbb{R}\mathbf{n})^{\bot}} are \ul{orthogonal compliments}, so:

\begin{itemize}

      \vItem
            \iMbox{\ds\mathrm{proj}_{L} = \widehat{\mathbf{n}}\widehat{\mathbf{n}}^{T}}
            is orthogonal projection \textbf{onto} \iMbox{L}
            \emph{(\textbf{along} \iMbox{P})}
      \vItem
            \iMbox{\ds\mathrm{proj}_{P} = \mathrm{id}_{\ds\mathbb{R}^{n}} - \mathrm{proj}_{L} = \mathbf{I}_{n}-\widehat{\mathbf{n}}\widehat{\mathbf{n}}^{T}}
            is orthogonal projection \textbf{onto} \iMbox{P} *(\textbf{along}
            \iMbox{L})
      \vItem
            \iMbox{\ds L = \mathrm{im}\left(\mathrm{proj}_{L}\right) = \ker\left(\mathrm{proj}_{P}\right)}
            and
            \iMbox{\ds P = \ker\left(\mathrm{proj}_{L}\right) = \mathrm{im}\left(\mathrm{proj}_{P}\right)}
      \vItem
            \iMbox{\ds\mathbb{R}^{n} = \mathbb{R}\mathbf{n} \oplus(\mathbb{R}\mathbf{n})^{\bot}},
            i.e.~all vectors \iMbox{\mathbf{v} \in \mathbb{R}^{n}} uniquely
            decomposed into
            \iMbox{\ds \mathbf{v} = \mathbf{v}_{L} + \mathbf{v}_{P}}
\end{itemize}


\subsection*{Householder Maps: reflections}

\begin{itemize}

      \vItem
            Two points \iMbox{\ds \mathbf{x},\mathbf{y} \in \mathbb{E}^{n}} are
            \textbf{reflections} w.r.t hyperplane
            \iMbox{\ds P = (\mathbb{R}\mathbf{n})^{\bot} + \mathbf{c}} if:

            \begin{enumerate}
                  \vItem 1)
                        The translation
                        \iMbox{\ds \overrightarrow{\mathbf{x}\mathbf{y}} = \mathbf{y} - \mathbf{x}}
                        is \textbf{\emph{parallel}} to normal \iMbox{\mathbf{n}},
                        i.e. \iMbox{\ds \overrightarrow{\mathbf{x}\mathbf{y}} = \lambda \mathbf{n}}
                  \vItem 2)
                        Midpoint \iMbox{\ds m = 1 / 2(\mathbf{x} + \mathbf{y}) \in P}
                        \textbf{\emph{lies}} on \iMbox{P},
                        i.e. \iMbox{m \cdot \mathbf{n} = \mathbf{c} \cdot \mathbf{n}}
            \end{enumerate}
      \vItem
            Suppose
            \iMbox{\ds P_{\mathbfit{u}} = (\mathbb{R}\mathbfit{u})^{\bot}}
            goes through the origin with unit normal \iMbox{\ds \mathbfit{u} \in \mathbb{R}^{n}}

            \begin{itemize}

                  \vItem
                        \textbf{Householder matrix}
                        \iMbox{\ds H_{\mathbfit{u}} = \mathbf{I}_{n}-2\mathbfit{u}\mathbfit{u}^{T}}
                        is reflection w.r.t. hyperplane \iMbox{\ds P_{\mathbfit{u}}}

                  \vItem
                        Recall: let
                        \iMbox{\ds L_{\mathbfit{u}} = \mathbb{R}\mathbfit{u}}

                        \tcbbreak

                        \begin{itemize}

                              \vItem
                                    \iMbox{\ds\mathrm{proj}_{L_{\mathbfit{u}}} = \mathbfit{u}\mathbfit{u}^{T}}
                                    and
                                    \iMbox{\ds\mathrm{proj}_{P_{\mathbfit{u}}} = \mathbf{I}_{n} - \mathbfit{u}\mathbfit{u}^{T}}
                                    =>
                                    \iMbox{\ds H_{\mathbfit{u}} = \mathrm{proj}_{P_{\mathbfit{u}}} - \mathrm{proj}_{L_{\mathbfit{u}}}}
                                    
                              \vItem
                                    \textbf{Visualize} as preserving component in
                                    \iMbox{\ds P_{\mathbfit{u}}}, then flipping component in \iMbox{\ds L_{\mathbfit{u}}}
                        \end{itemize}
                  \vItem
                        \iMbox{\ds H_{\mathbfit{u}}} is \ul{involutory, orthogonal and symmetric},
                        i.e. \iMbox{\ds H_{\mathbfit{u}} = H_{\mathbfit{u}}^{-1} = H_{\mathbfit{u}}^{T}}
            \end{itemize}
\end{itemize}