\subsection*{Thin QR Decomposition w/ Gram-Schmidt
  (GS)}

\begin{itemize}

  \item
        Consider \textbf{full-rank}
        \iMbox{\ds A = [\mathbf{a}_{1}|\dots|\mathbf{a}_{n}] \in \mathbb{R}^{m \times n}}
        \emph{(\iMbox{m \geq n})},
        i.e.~\iMbox{\ds\mathbf{a}_{1},\dots,\mathbf{a}_{n} \in \mathbb{R}^{m}}
        are linearly independent

        \begin{itemize}

          \item
                Apply {[}{[}tutorial 1\#Gram-Schmidt method to generate orthonormal
                basis from any linearly independent vectors\textbar GS{]}{]}
                \iMbox{\ds\mathbf{q}_{1},\dots,\mathbf{q}_{n} \leftarrow \mathrm{GS}(\mathbf{a}_{1},\dots,\mathbf{a}_{n})}
                to build \textbf{ONB}
                \iMbox{\ds\langle \mathbf{q}_{1},\dots,\mathbf{q}_{n} \rangle \in \mathbb{R}^m}
                for \iMbox{\mathrm{C}(A)}
          \item
                \textbf{For exams}: more efficient to compute as
                \iMbox{\ds \mathbf{u}_{j+1} = \mathbf{a}_{j+1} - Q_{j}\mathbf{c}_{j}}

                \begin{enumerate}
                  \def\labelenumi{\arabic{enumi})}

                  \item
                        Gather
                        \iMbox{\ds Q_{j} = [\mathbf{q}_{1}|\dots|\mathbf{q}_{j}] \in \mathbb{R}^{m \times j}}
                        \textbf{all-at-once}
                  \item
                        Compute
                        \iMbox{\ds \mathbf{c}_{j} = [\mathbf{q}_{1} \cdot \mathbf{a}_{j+1}, \dots, \mathbf{q}_{j} \cdot \mathbf{a}_{j+1}]^{T} \in \mathbb{R}^{j}}
                        \textbf{all-at-once}
                  \item
                        Compute \iMbox{\ds Q_{j}\mathbf{c}_{j} \in \mathbb{R}^{m}}, and
                        subtract from \iMbox{\mathbf{a}_{j+1}} \textbf{all-at-once}
                \end{enumerate}
          \item
                Can now rewrite
                \iMbox{\ds\mathbf{a}_{j}= \sum_{i=1}^{j} (\mathbf{q}_{i} \cdot \mathbf{a}_{j})\mathbf{q}_{i} = \mathrm{Q}_{j}\mathbf{c}_{j}}
        \end{itemize}
  \item
        Choose
        \iMbox{\ds \mathbf{Q} = Q_{n} = [\mathbf{q}_{1}|\dots|\mathbf{q}_{n}] \in \mathbb{R}^{m \times n}},
        notice its {[}{[}tutorial 1\#Orthogonality
        concepts\textbar semi-orthogonal{]}{]} since
        \iMbox{\ds \mathbf{Q}^{T} \mathbf{Q} = \mathbf{I}_{n}}

        \begin{itemize}

          \item
                Notice =\textgreater{}
                \iMbox{\ds\mathbf{a}_{j} = Q_{j} \mathbf{c}_{j} = \mathbf{Q}[\mathbf{q}_{1} \cdot \mathbf{a}_{j}, \dots, \mathbf{q}_{j} \cdot \mathbf{a}_{j}, 0, \dots, 0 ]^{T} = \mathbf{Q} \mathrm{r}_{j}}
          \item
                Let
                \iMbox{R = [\mathrm{r}_{1}|\dots|\mathrm{r}_{n}] \in \mathbb{R}^{n \times n}}
                =\textgreater{}
                \iMbox{\ds A = \mathbf{Q}R = \mathbf{Q} \begin{bmatrix}\mathbf{q}_{1}^{T} \mathbf{a}_{1} & \dots & \mathbf{q}_{1}^{T} \mathbf{a}_{n} \\ & \ddots & \vdots \\ 0 & & \mathbf{q}_{n}^{T} \mathbf{a}_{n}\end{bmatrix}},
                notice its {[}{[}tutorial 1\#Properties of
                matrices\textbar upper-triangular{]}{]}
        \end{itemize}
\end{itemize}

\subsection*{Full QR Decomposition}

\begin{itemize}

  \item
        Consider \textbf{full-rank}
        \iMbox{\ds A = [\mathbf{a}_{1}|\dots|\mathbf{a}_{n}] \in \mathbb{R}^{m \times n}}
        \emph{(\iMbox{m \geq n})},
        i.e.~\iMbox{\ds\mathbf{a}_{1},\dots,\mathbf{a}_{n} \in \mathbb{R}^{m}}
        are linearly independent
  \item
        Apply {[}{[}\#Thin QR Decomposition w/ Gram-Schmidt (GS)\textbar thin
        QR decomposition{]}{]} to obtain:

        \begin{itemize}

          \item
                ONB
                \iMbox{\ds\langle \mathbf{q}_{1},\dots,\mathbf{q}_{n} \rangle \in \mathbb{R}^m}
                for \iMbox{\mathrm{C}(A)}
          \item
                Semi-orthogonal
                \iMbox{\ds Q_{1}= [\mathbf{q}_{1}|\dots|\mathbf{q}_{n}] \in \mathbb{R}^{m \times n}}
                and upper-triangular \iMbox{R_{1} \in \mathbb{R}^{n \times n}},
                where \iMbox{A = Q_{1}R_{1}}
        \end{itemize}
  \item
        {[}{[}tutorial 3\#Tricks Computing orthonormal vector-set
        extensions\textbar Compute basis extension{]}{]} to obtain remaining
        \iMbox{\mathbf{q}_{n+1},\dots,\mathbf{q}_{m} \in \mathbb{R}^{m}},
        where \iMbox{\ds\langle \mathbf{q}_{1},\dots,\mathbf{q}_{m} \rangle}
        is \textbf{ONB} for \iMbox{\mathbb{R}^{m}}

        \begin{itemize}

          \item
                Notice
                \iMbox{\ds\langle \mathbf{q}_{n+1},\dots,\mathbf{q}_{m} \rangle} is
                \textbf{ONB} for \iMbox{\mathrm{C}(A)^{\perp} = \ker(A^{T})}
          \item
                Let
                \iMbox{Q_{2} = [\mathbf{q}_{n+1}|\dots|\mathbf{q}_{m}] \in \mathbb{R}^{m \times (m-n)}},
                let \iMbox{\ds Q = [Q_{1}|Q_{2}] \in \mathbb{R}^{m \times m}}, let
                \iMbox{R = [R_{1}; \mathbf{0}_{m-n}] \in \mathbb{R}^{m \times n}}
        \end{itemize}
  \item
        Then \textbf{full QR decomposition} is
        \iMbox{A = QR = [Q_{1}|Q_{2}]\begin{bmatrix}R_{1}\\ \mathbf{0}_{m-n}\end{bmatrix} = Q_{1}R_{1}}

        \begin{itemize}

          \item
                \iMbox{Q} is \textbf{orthogonal}, i.e.~\iMbox{Q^{-1} = Q^{T}}, so
                its a basis transformation
          \item
                \iMbox{\ds \mathrm{proj}_{\mathrm{C}(A)} = Q_{1}Q_{1}^{T}},
                \iMbox{\ds \mathrm{proj}_{\mathrm{C}(A)^{\perp}} = Q_{2}Q_{2}^{T}}
                are {[}{[}tutorial 1\#Projection properties\textbar orthogonal
                projections{]}{]} \textbf{onto} \iMbox{C(A)},
                \iMbox{\mathrm{C}(A)^{\perp} = \ker(A^{T})} \emph{respectively}
          \item
                Notice:
                \iMbox{\ds QQ^T = \mathbf{I}_{m} = Q_{1}Q_{1}^{T} + Q_{2}Q_{2}^{T}}
        \end{itemize}
  \item
        \textbf{Generalizable} to \iMbox{A \in \mathbb{C}^{m \times n}} by
        changing transpose to conjugate-transpose

        \begin{itemize}

          \item
                Inner product \iMbox{x^{T}y} =\textgreater{} \iMbox{x^{\dagger}y}
          \item
                Orthogonal matrix \iMbox{U^{-1} = U^{T}} =\textgreater{} unitary
                matrix \iMbox{U^{-1} = U^{\dagger}}

                \begin{itemize}

                  \item
                        For orthogonal
                        \iMbox{\ds U = [ \mathbf{u}_{1}|\dots|\mathbf{u}_{k} ] \in \mathbb{R}^{m \times k}}
                        =\textgreater{} \iMbox{\ds \mathrm{proj}_{U} = UU^T} projects
                        \textbf{\emph{onto}} \iMbox{\ds \mathrm{C}(U)}
                  \item
                        For unitary
                        \iMbox{\ds U = [ \mathbf{u}_{1}|\dots|\mathbf{u}_{k} ] \in \mathbb{C}^{m \times k}}
                        =\textgreater{} \iMbox{\ds \mathrm{proj}_{U} = UU ^{\dagger}}
                        projects \textbf{\emph{onto}} \iMbox{\ds \mathrm{C}(U)}
                \end{itemize}
          \item
                And so on\ldots{}
        \end{itemize}
\end{itemize}

\subsection*{\texorpdfstring{Lines and hyperplanes in Euclidean space
    \iMbox{\mathbb{E}^{n}({=}\mathbb{R}^{n})}}{Lines and hyperplanes in Euclidean space }}

\begin{itemize}

  \item
        Consider \textbf{standard Euclidean space}
        \iMbox{\mathbb{E}^{n}({=}\mathbb{R}^{n})}

        \begin{itemize}

          \item
                with standard basis
                \iMbox{\ds\langle \mathbf{e}_{1},\dots,\mathbf{e}_{n} \rangle \in \mathbb{R}^n}
          \item
                with standard origin \iMbox{\mathbf{0} \in \mathbb{R}^{n}}
        \end{itemize}
  \item
        A \textbf{line} \iMbox{\ds L = \mathbb{R}\mathbf{n} + \mathbf{c}} is
        \emph{characterized} by direction
        \iMbox{\mathbf{n} \in \mathbb{R}^{n}}
        \emph{(\iMbox{\mathbf{n} \neq \mathbf{0}})} and offset from origin
        \iMbox{\mathbf{c} \in L}

        \begin{itemize}

          \item
                It is customary that:

                \begin{itemize}

                  \item
                        \iMbox{\mathbf{n}} is a \textbf{unit vector},
                        i.e.~\iMbox{\ds \lVert \mathbf{n} \rVert = \lVert \widehat{\mathbf{n}} \rVert = 1}
                  \item
                        \iMbox{\mathbf{c} \in L} is \textbf{closest point to origin},
                        i.e.~\iMbox{\mathbf{c} \perp \mathbf{n}}
                \end{itemize}
          \item
                If \iMbox{\ds \mathbf{c} \neq \lambda \mathbf{n}} =\textgreater{}
                \iMbox{L} \textbf{not} vector-subspace of \iMbox{\mathbb{R}^{n}}

                \begin{itemize}

                  \item
                        i.e.~\iMbox{\mathbf{0} \not\in L}, i.e.~\iMbox{L} doesn't go
                        through the origin
                  \item
                        \iMbox{L} \textbf{is} affine-subspace of \iMbox{\mathbb{R}^{n}}
                \end{itemize}
          \item
                If \iMbox{\mathbf{c} = \lambda \mathbf{n}},
                i.e.~\iMbox{\ds L = \mathbb{R} \mathbf{n}} =\textgreater{}
                \iMbox{\ds L} \textbf{is} vector-subspace of \iMbox{\mathbb{R}^{n}}

                \begin{itemize}

                  \item
                        i.e.~\iMbox{\mathbf{0} \in L}, i.e.~\iMbox{L} goes through the
                        origin
                  \item
                        \iMbox{L} has \iMbox{\dim(L) = 1} and orthonormal basis (ONB)
                        \iMbox{\ds\{ \ \widehat{\mathbf{n}} \ \}}
                \end{itemize}
        \end{itemize}
  \item
        A \textbf{hyperplane}
        \iMbox{\ds \begin{align}P = (\mathbb{R}\mathbf{n})^{\bot} + \mathbf{c} &= \left\{ \ x + \mathbf{c} \ \middle| \ x \in \mathbb{R}^{n}, x \perp \mathbf{n} \ \right\} \\ &= \left\{ \ x \in \mathbb{R}^{n} \ \middle| \ x \cdot \mathbf{n} = \mathbf{c} \cdot \mathbf{n} \ \right\}\end{align}}
        is \emph{characterized} by normal
        \iMbox{\mathbf{n} \in \mathbb{R}^{n}}
        \emph{(\iMbox{\mathbf{n} \neq \mathbf{0}})} and offset from origin
        \iMbox{\mathbf{c} \in P}

        \begin{itemize}

          \item
                It \emph{represents} an \textbf{\iMbox{(n-1)}-dimensional slice} of
                the \textbf{\iMbox{n}-dimensional space}

                \begin{itemize}

                  \item
                        \textbf{Points} are hyperplanes for \iMbox{n=1}
                  \item
                        \textbf{Lines} are hyperplanes for \iMbox{n=2}
                  \item
                        \textbf{Planes} are hyperplanes for \iMbox{n=3}
                \end{itemize}
          \item
                It is customary that:

                \begin{itemize}

                  \item
                        \iMbox{\mathbf{n}} is a \textbf{unit vector},
                        i.e.~\iMbox{\ds \lVert \mathbf{n} \rVert = \lVert \widehat{\mathbf{n}} \rVert = 1}
                  \item
                        \iMbox{\mathbf{c} \in P} is \textbf{closest point to origin},
                        i.e.~\iMbox{\mathbf{c} = \lambda \mathbf{n}}
                  \item
                        With those =\textgreater{}
                        \iMbox{P = \left\{ \ x \in \mathbb{R}^{n} \ \middle| \ x \cdot \mathbf{n} = \lambda \ \right\}}
                \end{itemize}
          \item
                If \iMbox{\mathbf{c} \cdot \mathbf{n} \neq \mathbf{0}}
                =\textgreater{} \iMbox{P} \textbf{not} vector-subspace of
                \iMbox{\mathbb{R}^{n}}

                \begin{itemize}

                  \item
                        i.e.~\iMbox{\mathbf{0} \not\in P}, i.e.~\iMbox{P} doesn't go
                        through the origin
                  \item
                        \iMbox{P} \textbf{is} affine-subspace of \iMbox{\mathbb{R}^{n}}
                \end{itemize}
          \item
                If \iMbox{\mathbf{c} \cdot \mathbf{n} = \mathbf{0}},
                i.e.~\iMbox{\ds P = (\mathbb{R}\mathbf{n})^{\bot}} =\textgreater{}
                \iMbox{\ds P} \textbf{is} vector-subspace of \iMbox{\mathbb{R}^{n}}

                \begin{itemize}

                  \item
                        i.e.~\iMbox{\mathbf{0} \in P}, i.e.~\iMbox{P} goes through the
                        origin
                  \item
                        \iMbox{P} has \iMbox{\dim(P) = n-1}
                \end{itemize}
        \end{itemize}
  \item
        Notice \iMbox{L = \mathbb{R}\mathbf{n}} and
        \iMbox{P = (\mathbb{R}\mathbf{n})^{\bot}} are orthogonal compliments,
        so:

        \begin{itemize}

          \item
                \iMbox{\ds\mathrm{proj}_{L} = \widehat{\mathbf{n}}\widehat{\mathbf{n}}^{T}}
                is orthogonal projection \textbf{onto} \iMbox{L}
                \emph{(\textbf{along} \iMbox{P})}
          \item
                \iMbox{\ds\mathrm{proj}_{P} = \mathrm{id}_{\ds\mathbb{R}^{n}} - \mathrm{proj}_{L} = \mathbf{I}_{n}-\widehat{\mathbf{n}}\widehat{\mathbf{n}}^{T}}
                is orthogonal projection \textbf{onto} \iMbox{P} *(\textbf{along}
                \iMbox{L})
          \item
                \iMbox{\ds L = \mathrm{im}\left(\mathrm{proj}_{L}\right) = \ker\left(\mathrm{proj}_{P}\right)}
                and
                \iMbox{\ds P = \ker\left(\mathrm{proj}_{L}\right) = \mathrm{im}\left(\mathrm{proj}_{P}\right)}
          \item
                \iMbox{\ds\mathbb{R}^{n} = \mathbb{R}\mathbf{n} \oplus(\mathbb{R}\mathbf{n})^{\bot}},
                i.e.~all vectors \iMbox{\mathbf{v} \in \mathbb{R}^{n}} uniquely
                decomposed into
                \iMbox{\ds \mathbf{v} = \mathbf{v}_{L} + \mathbf{v}_{P}}
        \end{itemize}
\end{itemize}

\subsection*{Reflection w.r.t. hyperplanes and Householder
  Maps}

\begin{itemize}

  \item
        Two points \iMbox{\ds \mathbf{x},\mathbf{y} \in \mathbb{E}^{n}} are
        \textbf{reflections} w.r.t hyperplane
        \iMbox{\ds P = (\mathbb{R}\mathbf{n})^{\bot} + \mathbf{c}} if:

        \begin{enumerate}
          \def\labelenumi{\arabic{enumi})}

          \item
                The translation
                \iMbox{\ds \overrightarrow{\mathbf{x}\mathbf{y}} = \mathbf{y} - \mathbf{x}}
                is \textbf{\emph{parallel}} to normal \iMbox{\mathbf{n}},
                i.e.~\iMbox{\ds \overrightarrow{\mathbf{x}\mathbf{y}} = \lambda \mathbf{n}}
          \item
                Midpoint \iMbox{\ds m = 1 / 2(\mathbf{x} + \mathbf{y}) \in P}
                \textbf{\emph{lies}} on \iMbox{P},
                i.e.~\iMbox{m \cdot \mathbf{n} = \mathbf{c} \cdot \mathbf{n}}
        \end{enumerate}
  \item
        Suppose
        \iMbox{\ds P_{\boldsymbol{u}} = (\mathbb{R}\boldsymbol{u})^{\bot}}
        goes through the origin with unit normal
        \iMbox{\ds \boldsymbol{u} \in \mathbb{R}^{n}}

        \begin{itemize}

          \item
                \textbf{Householder matrix}
                \iMbox{\ds H_{\boldsymbol{u}} = \mathbf{I}_{n}-2\boldsymbol{u}\boldsymbol{u}^{T}}
                is reflection w.r.t. hyperplane \iMbox{\ds P_{\boldsymbol{u}}}
          \item
                Recall: let
                \iMbox{\ds L_{\boldsymbol{u}} = \mathbb{R}\boldsymbol{u}}

                \begin{itemize}

                  \item
                        \iMbox{\ds\mathrm{proj}_{L_{\boldsymbol{u}}} = \boldsymbol{u}\boldsymbol{u}^{T}}
                        and
                        \iMbox{\ds\mathrm{proj}_{P_{\boldsymbol{u}}} = \mathbf{I}_{n} - \boldsymbol{u}\boldsymbol{u}^{T}}
                        =\textgreater{}
                        \iMbox{\ds H_{\boldsymbol{u}} = \mathrm{proj}_{P_{\boldsymbol{u}}} - \mathrm{proj}_{L_{\boldsymbol{u}}}}
                  \item
                        \textbf{Visualize} as preserving component in
                        \iMbox{\ds P_{\boldsymbol{u}}}, then flipping component in
                        \iMbox{\ds L_{\boldsymbol{u}}}
                \end{itemize}
          \item
                \iMbox{\ds H_{\boldsymbol{u}}} is involutory, orthogonal and
                symmetric,
                i.e.~\iMbox{\ds H_{\boldsymbol{u}} = \ds H_{\boldsymbol{u}}^{-1} = \ds H_{\boldsymbol{u}}^{T}}
        \end{itemize}
\end{itemize}