\mkImg{image copy 13}
\mkImg{image copy 12}

\textcolor{cornellred}{\textbf{Goals of Scheduling Algorithms}} => 
\textbf{Ensure fairness}: Comparable processes should get comparable services;
\textbf{Avoid indefinite postponement}: No process should starve;
\textbf{Enforce policy}: E.g. priorities;
\textbf{Maximize resource utilization}: CPU, I/O devices;
\textbf{Minimize overhead}: From context switches, scheduling decisions

\textcolor{cornellred}{\textbf{For Batch Systems}} => 
\textbf{Throughput}: Jobs per unit of time;
\textbf{Turnaround time}: Time between job submission \& completion


\textcolor{cornellred}{\textbf{For Interactive Systems}} => 
\textbf{Response time}: Time between request issued and first response

\mkImg{image copy 14}

\hSep

\mkImg{image copy 15}
\^{} Good avg. turnaround time when different runtimes.

\textit{\textbf{RR quantum (time slice)}} => should be larger than context switch cost,
but still provide decent response time;
\textit{\textbf{RR overhead}} =>
\(\textrm{RR quantum} / (\textrm{RR quantum} + \textrm{ctx. switch time})\).

\mkImg{image copy 16}
\mkImg{image copy 17}
\mkImg{image copy 18}
\^{} within each user can use RR to schedule.

\mkImg{image copy 19}

\textbf{MULTILEVEL FEEDBACK QUEUES} — Form of priority
scheduling. One queue for each priority level. Within each
queue, can use diff. scheduling (usually RR). Need to
determine current nature of job, worry about starvation of
lower-priority jobs. \textit{Feedback} comes from periodically
recomputing priorities => e.g. based on how much CPU they
used recently, increase job's priority as it ages, exponentially weighed
moving average. Not very flexible, does not react quickly to changes, cheating is a concern.

\mkImg{image copy 20}

\hSep