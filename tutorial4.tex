\subsection*{Cholesky Decomposition}


Consider \textbf{positive (semi-)definite}
\iMbox{A \in \mathbb{R}^{n \times n}}

\textbf{Cholesky Decomposition} is \iMbox{A = LL^{T}} where \iMbox{L}
is lower-triangular
\begin{itemize}

      \vItem
            For positive semi-definite => \textbf{always exists},
            but \textbf{non-unique}
      \vItem
            For positive-definite => \textbf{always \emph{uniquely}
                  exists} s.t. diagonals of \iMbox{L} are positive
\end{itemize}

\hSep % ---

Finding a Cholesky Decomposition:
\begin{itemize}

      \vItem
            Compute \iMbox{LL^{T}} and solve \iMbox{A = LL^T} by matching terms
      \vItem
            For square roots always pick positive
      \vItem
            If there is \textbf{exact solution} then \textbf{positive-definite}
      \vItem
            If there are \textbf{free variables} at the end, then
            \textbf{positive semi-definite}

            \begin{itemize}

                  \vItem
                        i.e.~the decomposition is a \textbf{solution-set} parameterized on
                        \textbf{free variables}
                  \vItem
                        e.g.~\iMbox{\begin{bmatrix}1&1&1\\1&1&1\\1&1&2\end{bmatrix} = LL^{T}}
                        where
                        \iMbox{L = \begin{bmatrix}1&0&0\\1&0&0\\1&c&\sqrt{ 1-c^2 }\end{bmatrix}, c \in [0,1]}
            \end{itemize}
\end{itemize}

\hSep % ---

If \iMbox{A = LL^T} you can use \underline{forward/backward substitution} to
\textbf{solve equations}

\begin{itemize}

      \vItem
            For \iMbox{Ax = b} => let \iMbox{y = L^Tx}
      \vItem
            Solve \iMbox{Ly = b} by forward substitution to \textbf{find
                  \iMbox{y}}
      \vItem
            Solve \iMbox{L^{T}x = y} by backward substitution to \textbf{find
                  \iMbox{x}}
\end{itemize}

For \iMbox{n=3} =>
\iMbox{L = \begin{bmatrix} l_{11} & 0 & 0 \\ l_{21} & l_{22} & 0 \\ l_{31} & l_{32} & l_{33}\end{bmatrix}},
\iMbox{LL^{T} = \begin{bmatrix} l_{11}^2 & l_{11} l_{21} & l_{11} l_{31} \\ l_{11} l_{21} & l_{21}^2+l_{22}^2 
      & l_{21} l_{31}+l_{22} l_{32} \\ l_{11} l_{31} & l_{21} l_{31}+l_{22} l_{32} & l_{31}^2+l_{32}^2+l_{33}^2\end{bmatrix}}

\subsection*{Forward/backward substitution}

\begin{itemize}

      \vItem
            \textbf{Forward substitution}: for lower-triangular
            \iMbox{L = \begin{bmatrix}\ell_{1,1} & & 0 \\ \vdots & \ddots & \\ \ell_{n, 1} & \ldots & \ell_{n, n}\end{bmatrix}}

            \begin{itemize}

                  \vItem
                        For \iMbox{Lx = b}, just \textbf{solve} the first row
                        \iMbox{\ell_{1,1} x_{1} = b_{1} \implies x_{1} = \frac{b_{1}}{\ell_{1,1}}}
                        and \textbf{substitute down}
                  \vItem
                        Then \textbf{solve} the second row
                        \iMbox{
                              \ell_{2,1}x_{1} + \ell_{2,2}x_{2} = b_{2} \implies x_{2} = \frac{b_{2} - \ell_{2,1}x_{1}}{\ell_{2,2}}
                        }
                        and \textbf{substitute down}
                  \vItem
                        \ldots and so on until all \iMbox{x_{i}} are solved
            \end{itemize}
      \vItem
            \textbf{Backward substitution}: for upper-triangular
            \iMbox{U = \begin{bmatrix}u_{1,1} & \dots & u_{1,n} \\ & \ddots & \vdots \\ 0 & & u_{n, n}\end{bmatrix}}

            \begin{itemize}

                  \vItem
                        For \iMbox{Ux = b}, just \textbf{solve} the last row
                        \iMbox{u_{n,n} x_{n} = b_{n} \implies x_{n} = \frac{b_{n}}{u_{n,n}}}
                        and \textbf{substitute up}
                  \vItem
                        Then \textbf{solve} the second-to-last row
                        \iMbox{
                              \begin{aligned}
                                    &u_{n-1,n-1}x_{n-1} + u_{n-1,n}x_{n} = b_{n-1} \\ &\implies x_{n-1} = \frac{b_{n-1} - u_{n-1,n-1}x_{n-1}}{u_{n-1,n}}
                              \end{aligned}
                        }
                        and \textbf{substitute up}
                  \vItem
                        \ldots and so on until all \iMbox{x_{i}} are solved
            \end{itemize}
\end{itemize}