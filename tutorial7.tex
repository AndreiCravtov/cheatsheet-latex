\subsection*{Elementary Matrices}

\begin{itemize}

  \item
        Identity
        \iMbox{\ds \mathbf{I}_{n} = [\mathbf{e}_{1}|\dots|\mathbf{e}_{n}] = [\mathbf{e}_{1};\dots;\mathbf{e}_{n}]}
        has elementary vectors \iMbox{\ds\mathbf{e}_{1},\dots,\mathbf{e}_{n}}
        for rows/columns
  \item
        \textbf{Row/column switching}: permutation matrix \iMbox{\ds P_{ij}}
        obtained by switching \iMbox{\ds\mathbf{e}_{i}} and
        \iMbox{\mathbf{e}_{j}} in \iMbox{\mathbf{I}_{n}} \emph{(same for
          rows/columns)}

        \begin{itemize}

          \item
                Applying \iMbox{\ds P_{ij}} \textbf{from left} will switch rows,
                \textbf{from right} will swap columns
          \item
                \iMbox{\ds P_{ij} = P_{ij}^{T} = P_{ij}^{-1}}, i.e.~applying twice
                will \textbf{undo} it
        \end{itemize}
  \item
        \textbf{Row/column scaling}: \iMbox{\ds D_{i}(\lambda)} obtained by
        scaling \iMbox{\ds\mathbf{e}_{i}} by \iMbox{\lambda} in
        \iMbox{\mathbf{I}_{n}} \emph{(same for rows/columns)}

        \begin{itemize}

          \item
                Applying \iMbox{\ds P_{ij}} \textbf{from left} will scale rows,
                \textbf{from right} will scale columns
          \item
                \iMbox{\ds D_{i}(\lambda) = \mathrm{diag}(1,\dots,\lambda,\dots,1)}
                so all \textbf{diagonal} properties apply,
                e.g.~\iMbox{\ds D_{i}(\lambda)^{-1} = D_{i}(\lambda^{-1})}
        \end{itemize}
  \item
        \textbf{Row addition}:
        \iMbox{\ds L_{ij}(\lambda) = \mathbf{I}_{n} + \lambda\mathbf{e}_{i}\mathbf{e}_{j}^{T}}
        performs \iMbox{\ds R_{i} \leftarrow R_{i} + \lambda R_{j}} when
        applying \textbf{from left}

        \begin{itemize}

          \item
                \iMbox{\ds\lambda\mathbf{e}_{i}\mathbf{e}_{j}^{T}} is zeros except
                for \textbf{\iMbox{\lambda} in \iMbox{\ds(i,j)}-th entry}
          \item
                \iMbox{L_{ij}(\lambda)^{-1} = L_{ij}(-\lambda)} both triangular
                matrices
        \end{itemize}
\end{itemize}

\subsection*{LU factorization w/ Gaussian
  elimination}

\begin{itemize}

  \item
        {[}{[}tutorial 1\#Representing EROs/ECOs as transformation
        matrices\textbar Recall that{]}{]} you can represent \textbf{EROs} and
        \textbf{ECOs} as transformation matrices \iMbox{R,C}
        \emph{respectively}
  \item
        \textbf{\iMbox{LU} factorization} =\textgreater{} finds \iMbox{A = LU}
        where \iMbox{L,U} are lower/upper triangular \emph{respectively}
  \item
        \textbf{Naive Gaussian Elimination} performs
        \iMbox{\ds [I_{m} \ | \ A \ | \ I_{n}] \rightsquigarrow [R^{-1} \ | \ U \ | \ I_{n}]}
        to get \iMbox{AI_{n} = R^{-1}U} using only row addition

        \begin{itemize}

          \item
                \iMbox{R^{-1}}, i.e.~\textbf{inverse EROs} in reversed order, is
                \textbf{lower-triangular} so \iMbox{L = R^{-1}}
          \item
                !{[}{[}Pasted image 20250419051217.png\textbar400{]}{]}
          \item
                The \textbf{pivot element} is simply diagonal entry
                \iMbox{\ds u_{kk}^{(k-1)}}; fails if
                \iMbox{\ds u_{kk}^{(k-1)} \approx 0}
          \item
                \iMbox{\ds\tilde{L} \tilde{U}=A+\delta A},
                \iMbox{\ds \frac{\|\delta A\|}{\|L\| \cdot\|U\|}= O\left(\epsilon_{ \mathrm{mach}}\right)};
                only \textbf{backwards stable} if
                \iMbox{\|L\| \cdot\|U\| \approx\|A\|}
          \item
                Work required: \iMbox{{\sim}\frac{2}{3} m^3} flops
                \iMbox{{\sim}O\left(m^3\right)}
          \item
                Solving \iMbox{Ax = LUx} is \iMbox{{\sim}\frac{2}{3} m^3} flops
                \emph{(back substitution is \iMbox{O(m^2)})}
          \item
                \textbf{NOTE:} Householder triangularisation requires
                \iMbox{{\sim}\frac{4}{3} m^3}
        \end{itemize}
  \item
        \textbf{Partial pivoting} computes \iMbox{PA = LU} where \iMbox{P} is
        a permutation matrix =\textgreater{} \iMbox{PP^{T} = I}, i.e.~its
        orthogonal

        \begin{itemize}

          \item
                For each column \iMbox{j}, finds largest entry and row-swaps to make
                it new pivot =\textgreater{} \iMbox{\ds P_{j}}
          \item
                Then performs normal elimination on that column =\textgreater{}
                \iMbox{\ds L_{j}}
          \item
                Result is \iMbox{\ds L_{m-1} P_{m-1} \ldots L_2 P_2 L_1 P_1 A = U},
                where
                \iMbox{\ds L_{m-1} P_{m-1} \ldots L_2 P_2 L_1 P_1=L_{m-1}^{\prime} \ldots L_1^{\prime} P_{m-1} \ldots P_1}
          \item
                Setting
                \iMbox{\ds L=\left(L_{m-1}^{\prime} \ldots L_1^{\prime}\right)^{-1}},
                \iMbox{\ds P=P_{m-1} \ldots P_1} gives \iMbox{P A=L U}
          \item
                !{[}{[}Pasted image 20250420092322.png\textbar450{]}{]}
          \item
                Work required: \iMbox{{\sim}\frac{2}{3} m^3} flops
                \iMbox{{\sim}O\left(m^3\right)}; results in
                \iMbox{\ds L_{ij} \leq 1} so \iMbox{\ds \lVert L \rVert = O(1)}
          \item
                Stability depends on \textbf{growth-factor}
                \iMbox{\ds\rho=\frac{\max _{i, j}\left|u_{i, j}\right|}{\max _{i, j}\left|a_{i, j}\right|}}
                =\textgreater{} for partial pivoting \iMbox{\rho \leq 2^{m-1}}
          \item
                \iMbox{\lVert U \rVert = O(\rho \lVert A \rVert)} =\textgreater{}
                \iMbox{\ds \tilde{L} \tilde{U}=\tilde{P} A+\delta A},
                \iMbox{\ds\frac{\|\delta A\|}{\|A\|}=O\left(\rho \epsilon_{\text {machine }}\right)}
                =\textgreater{} only \textbf{backwards stable} if \iMbox{\rho=O(1)}
        \end{itemize}
  \item
        \textbf{Full pivoting} is \iMbox{PAQ = LU} finds largest entry in
        \textbf{bottom-right submatrix}

        \begin{itemize}

          \item
                Makes it \textbf{pivot} with row/column swaps before normal
                elimination
          \item
                Very expensive \iMbox{O(m^3)} search-ops, \textbf{partial pivoting}
                only needs \iMbox{O(m^2)}
        \end{itemize}
\end{itemize}

\subsection*{Systems of Equations: Iterative
  Techniques}

\begin{itemize}

  \item
        Let \iMbox{A,R,G \in \mathbb{R}^{n \times n}} where \iMbox{G^{-1}}
        exists =\textgreater{} \textbf{splitting} \iMbox{A = G + R} helps
        iteration

        \begin{itemize}

          \item
                \iMbox{A\mathbf{x}=\mathbf{b}} rewritten as
                \iMbox{\mathbf{x} = M \mathbf{x} + \mathbf{c}} where
                \iMbox{M = - G^{-1}R; \ \mathbf{c} = -G^{-1}\mathbf{b}}
          \item
                Define \iMbox{\ds f(\mathbf{x}) = M \mathbf{x} + \mathbf{c}} and
                sequence
                \iMbox{\ds \mathbf{x}^{(k+1)} = f(\mathbf{x}^{(k)}) = M \mathbf{x}^{(k)} + \mathbf{c}}
                with starting point \iMbox{\ds\mathbf{x}^{(0)}}
          \item
                \textbf{Limit} of \iMbox{\ds\langle\mathbf{x}_{k}\rangle} is fixed
                point of \iMbox{f} =\textgreater{} unique fixed point of \iMbox{f}
                is \textbf{solution} to \iMbox{A\mathbf{x}=\mathbf{b}}
          \item
                If \iMbox{\lVert - \rVert} is consistent norm and
                \iMbox{\lVert M \rVert < 1} then
                \iMbox{\ds\langle\mathbf{x}_{k}\rangle} converges for any
                \iMbox{\ds\mathbf{x}^{(0)}} \emph{(because Cauchy-completeness)}

                \begin{itemize}

                  \item
                        For splitting, we want \iMbox{\lVert M \rVert < 1} and easy to
                        compute \iMbox{M; \mathbf{c}}
                  \item
                        \textbf{Stopping criterion} usually the relative residual
                        \iMbox{\ds\frac{\left\|\mathbf{b}-A \mathbf{x}^{(k)}\right\|}{\|\mathbf{b}\|} \leq \epsilon}
                \end{itemize}
        \end{itemize}
  \item
        Assume \iMbox{A}'s \textbf{diagonal is non-zero} \emph{(w.l.o.g.
          permute/change basis if isn't)} then \iMbox{A = D + L + U}

        \begin{itemize}

          \item
                Where \iMbox{D} is \textbf{diagonal} of \iMbox{A}, \iMbox{L,U} are
                strict \textbf{lower/upper triangular} parts of \iMbox{A}
        \end{itemize}
  \item
        \textbf{Jacobi Method}: \iMbox{G = D;R = L + U} =\textgreater{}
        \iMbox{M = -D^{-1}(L + U); \mathbf{c} = D^{-1}\mathbf{b}}

        \begin{itemize}

          \item
                \iMbox{\ds \mathbf{x}_i^{(k+1)}=\frac{1}{A_{ii}}\left(\mathbf{b}_i-\sum_{j \neq i} A_{ij} \mathbf{x}_j^{(k)}\right)}
                =\textgreater{} \iMbox{\ds \mathbf{x}_i^{(k+1)}} only needs
                \iMbox{\ds\mathbf{b}_i; \ \mathbf{x}^{(k)}; \ A_{i \ast}}
                =\textgreater{} row-wise parallelization
        \end{itemize}
  \item
        \textbf{Gauss-Seidel (G-S) Method}: \iMbox{G = D + L;R = U}
        =\textgreater{}
        \iMbox{M = -(D + L)^{-1}U; \mathbf{c} = (D + L)^{-1}\mathbf{b}}

        \begin{itemize}

          \item
                \iMbox{\ds \mathbf{x}_i^{(k+1)}=\frac{1}{A_{ii}}\left(\mathbf{b}_i - \sum_{j=1}^{i-1} A_{ij} \mathbf{x}_j^{(k+1)}-\sum_{j=i+1}^n A_{ij} \mathbf{x}_j^{(k)}\right)}
          \item
                Computing \iMbox{\ds \mathbf{x}_i^{(k+1)}} needs
                \iMbox{\ds\mathbf{b}_i; \ \mathbf{x}^{(k)}; \ A_{i \ast}} and
                \iMbox{\ds \mathbf{x}_{j}^{(k+1)}} for \iMbox{j<i} =\textgreater{}
                lower storage requirements
        \end{itemize}
  \item
        \textbf{Successive over-relaxation (SOR)}:
        \iMbox{G = \omega^{-1} D + L;R = (1-\omega^{-1})D + U} =\textgreater{}
        \iMbox{M = - (\omega^{-1} D + L)^{-1}((1-\omega^{-1})D + U); \ \mathbf{c} = -(\omega^{-1} D + L)^{-1}\mathbf{b}}

        \begin{itemize}

          \item
                \iMbox{\ds \mathbf{x}_i^{(k+1)}=\frac{\omega}{A_{ii}}\left(\mathbf{b}_i-\sum_{j=1}^{i-1} A_{ij} \mathbf{x}_j^{(k+1)}-\sum_{j=i+1}^n A_{ij} \mathbf{x}_j^{(k)}\right)+(1-\omega) \mathbf{x}_i^{(k)}}
                for \textbf{relaxation factor} \iMbox{\omega>1}\\
        \end{itemize}
  \item
        If \iMbox{A} is \textbf{strictly row diagonally dominant} then
        Jacobi/Gauss-Seidel methods converge

        \begin{itemize}

          \item
                \iMbox{A} is \textbf{strictly row diagonally dominant} if
                \iMbox{\ds\left|A_{ii}\right|>\sum_{j \neq i}\left|A_{ij}\right|}
        \end{itemize}
  \item
        If \iMbox{A} is positive-definite then \textbf{G-S} and \textbf{SOR}
        (\iMbox{\omega \in (0,2)}) converge
\end{itemize}

\subsection*{Break up matrices into (uneven
  blocks)}

\begin{itemize}

  \item
        e.g.~symmetric \iMbox{A \in \mathbb{R}^{n \times n }} can become
        \iMbox{A=\left[\begin{array}{c|c}a_{1,1} & b \\ \hline b^T & C\end{array}\right]},
        then perform proofs on that
\end{itemize}

\subsection*{Catchup: metric spaces and
  limits}

\begin{itemize}

  \item
        Metrics obey these axioms

        \begin{itemize}

          \item
                \iMbox{d(x,x) = 0}
          \item
                \iMbox{x \neq y \implies d(x,y) > 0}
          \item
                \iMbox{d(x,y) = d(y,x)}
          \item
                \iMbox{d(x,z) \leq d(x,y) + d(y,z)}
        \end{itemize}
  \item
        For metric spaces, \textbf{mix-and-match} these infinite/finite limit
        definitions:

        \begin{itemize}

          \item
                \iMbox{\ds \lim_{ x \to + \infty } f(x) = +\infty \iff \forall r \in \mathbb{R}, \exists N \in \mathbb{N}, \forall x > N: \ \  f(x)>r}
          \item
                \iMbox{\ds \lim_{ x \to p } f(x) = L \iff \forall \varepsilon >0,\exists \delta > 0, \forall x \in A: \ \ 0 < d_{X}(x,p) < \delta \implies d_{Y}(f(x),L) < \varepsilon}
          \item
                \textbf{Cauchy sequences},
                i.e.~\iMbox{\ds\forall \varepsilon >0, \exists N \in \mathbb{N}, \forall m,n\geq N: \ \ d(a_{m}, a_{n})<\varepsilon},
                converge in \textbf{complete spaces}
        \end{itemize}
  \item
        You can manipulate matrix limits much \textbf{like in real analysis},
        e.g.~\iMbox{\ds \lim_{ n \to \infty }(A^{n}B+C) = \left(\lim_{ n \to \infty }A^{n} \right)B+C}
  \item
        Turn \textbf{metric limit} \iMbox{\ds\lim_{ n \to \infty } x_{n} = L}
        into \textbf{real limit}
        \iMbox{\ds\lim_{ n \to \infty } d(x_{n},L) = 0} to leverage real
        analysis

        \begin{itemize}

          \item
                Bounded \textbf{monotone sequences} converge in \iMbox{\mathbb{R}}
          \item
                Sandwich theorem for limits in \iMbox{\mathbb{R}} =\textgreater{}
                pick easy upper/lower bounds
          \item
                \iMbox{\ds\lim_{ n \to \infty }  r^n = 0 \iff |r| < 1} and
                \iMbox{\ds \lim_{ n \to \infty } \sum_{i=0}^{n} ar^{i} = \frac{a}{1-r} \iff |r| < 1}
        \end{itemize}
\end{itemize}